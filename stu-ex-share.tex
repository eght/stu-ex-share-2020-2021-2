%%% 8653513F-1D36-415A-AB62-83899679B30F.tex ---
%%
%% Filename: 8653513F-1D36-415A-AB62-83899679B30F.tex
%% Description: rpd-001
%% usage:
%%        0. C:\cygwin64\bin\mintty.exe -i /Cygwin-Terminal.ico -
%%        0. "C:\Program Files (x86)\Everything\Everything.exe"
%%        1. cd /cygdrive/f/git/secret/o/oo/oop/oopc/001/
%%        2. /cygdrive/c/texlive/2020/bin/win32/xelatex.exe 8653513F-1D36-415A-AB62-83899679B30F.tex
%%        3. /cygdrive/c/texlive/2020/bin/win32/bibtex.exe 8653513F-1D36-415A-AB62-83899679B30F.aux
%% Author: Gao, LiHui
%% Maintainer:
%% Created: Sat Jan 12 16:17:52 2019 (+0800)
%% Version:
%% Package-Requires: ()
%% Last-Updated: Mon Jan  4 10:47:53 2021 (+0800)
%%           By: Administrator
%%     Update #: 2425
%% URL:
%% Doc URL:
%% Keywords:
%% Compatibility:
%%%%%%%%%%%%%%%%%%%%%%%%%%%%%%% setting-001
%!TEX TS-program = xelatex
%!TEX encoding = UTF-8 Unicode
%%%%%%%%%%%%%%%%%%%%%%%%%%%%%%%% setting-002
% 中英文论文模板
% 对xelatex的中英文设置不同的字体
% XeTeX处理中文非常方便,不需要任何设置,
% 就能够使用系统中安装的TrueType和OpenType字体
% [website][https://www.cnblogs.com/tenderwx/p/6920283.html]
% date: Sun Nov 04 21:39:43 China Standard Time 2018
% version: 1.00
% effect: article
% usage: XeLaTeX -> BiBTeB -> XeLaTeX
%%%%%%%%%%%%%%%%%
% text editor:
% WinEdt 10.2
% Build: 20170413  (v. 10.2) - 64-bit
%%%%%%%%%%%%%%%%%
% This is XeTeX, Version 3.14159265-2.6-0.99999
% (TeX Live 2018/W32TeX)
% (preloaded format=xelatex)
%%%%%%%%%%%%%%%%%%%%%%%% setting-003
% 下面是常用字体名称:
% %%%%%
% STCaiyun,华文彩云:style=Regular
% YouYuan,幼圆:style=Regular
% STHupo,华文琥珀:style=Regular
% FZYaoTi,方正姚体:style=Regular
% NSimSun,新宋体:style=Regular
% FangSong,仿宋:style=Regular,Normal,…
% KaiTi,楷体:style=Regular,Normal,obyčejné,Standard,...
% Microsoft YaHei,微软雅黑:style=Regular,Normal,obyčejné,…
% SimSun,宋体:style=Regular
% STFangsong,华文仿宋:style=Regular
% STXinwei,华文新魏:style=Regular
% Arial Unicode MS:style=Regular,Normal,obyčejné,Standard,…
% STXingkai,华文行楷:style=Regular
% STLiti,华文隶书:style=Regular
% SimHei,黑体:style=Regular,Normal,…
% Adobe Heiti Std R,Adobe Heiti Std,Adobe 黑体 Std,
%             Adobe 黑体 Std R:style=Regular,R
%%
% STZhongsong,华文中宋:style=Regular
% Microsoft YaHei,微软雅黑:style=Bold,Negreta,tučné,fed,Fett,…
% FZShuTi,方正舒体:style=Regular
% Adobe Song Std L,Adobe Song Std,Adobe 宋体 Std,
%                  Adobe 宋体 Std L:style=Regular,L
%%
% STXihei,华文细黑:style=Regular
% LiSu,隶书:style=Regular
% STKaiti,华文楷体:style=Regular
%%%%%%%%%%%%%%%%%%%%%%%%%%%% setting-004
\documentclass[a4paper,10pt]{article} %% for draft
%\documentclass[a4paper,12pt]{article}
\usepackage{xeCJK}
\usepackage{fontspec}
\usepackage{amsmath}
%\setCJKmainfont{Microsoft YaHei} %% 设置CJK主字体,也就是设置\rmfamily的CJK字体
%\setCJKsansfont{} %% 设置CJK无衬线的字体,也就是设置\sffamily的CJK字体
%\setCJKmonofont{WenQuanYi Micro Hei Mono} %% 设置CJK的等宽字体,也就是设置\ttfamily的CJK字体
%\setCJKmainfont{KaiTi}
%\setCJKmainfont{FangSong}
%\setCJKmonofont{SimSun}
%\setmainfont{DejaVu Sans Mono}
%%%%%%%%%……剩下的包,随你加
\usepackage{amsmath,amssymb}
\usepackage{hyperref}
\usepackage{graphicx} %% insert picture
\usepackage{color,xcolor} %% 字体颜色
\usepackage{lineno} %% 加行号,添加lineno包
\usepackage{listings} %% 加行号
\usepackage{bm} %% bold face vector
\lstset{breaklines=true}
%\lstset{numbers=left, numberstyle=\scriptsize\ttfamily, numbersep=10pt, captionpos=b}
%\lstset{backgroundcolor=\color{gray-5}}
%\lstset{basicstyle=\small\ttfamily}
\lstset{framesep=4pt}
%%%%%%%%%%%%%%%%%%%%%% setting-005
\usepackage{hyperref} % Package for hyperlinks
%UTF-8 编码,Xe\LaTeX{} 编译。
%%%%%%%%%%%%%%%%%%%% setting-006
% \numberwithin{equation}{section} %% 公式按章节编号
%%%%%%%%%%%%%%%%%%% setting-007
% \usepackage{geometry} %% 调整页边距
% \geometry{a4paper,left=5cm,right=5cm,top=3cm,bottom=3cm} %% 单独调整
% \geometry{a4paper,left=4.5cm,right=4.5cm,top=2.5cm,bottom=2.5cm} %% 单独调整
% \geometry{a4paper,left=3.5cm,right=3.5cm,top=2.5cm,bottom=2.5cm} %% 单独调整
% \geometry{a4paper,scale=0.8} %% 整体调整
%%%%%%%%%%%%%%%%%%%% setting-008
\usepackage{authblk} %% 定义多个作者,地址,通讯邮箱
%%%%%%%%%%%%%%%%%%%% setting-009
%%%%
 \usepackage{showkeys} %% 标签太多了,都增知道哪个是哪个了,显示标签名就清楚了
% \usepackage[notref,notcite]{showkeys} %% 取消引用处的标签名
\usepackage[marginal]{showlabels}
%%%%
%%%%%%%%%%%%%%%%%%%% setting-010
%%%%
\usepackage[dvips=false,pdftex=false,vtex=false]{geometry}
\geometry{
%  paperwidth=6in,
%  paperheight=9in,
%  margin=2em,
%  bottom=1.5em,
%  nohead
} %% a4 size = 8.3 in x 11.7 in, Letter size = 8.5in x 11in, 1in = 2.54cm
%% %% %% %% \usepackage[cam,a4,center,dvips]{crop} %% 裁切标记,套准标记,颜色条,页面信息, 裁剪标记Crop Mark, 就是给印刷厂裁剪纸张用的.
\usepackage[cam,a4,center,dvips]{crop} %% 裁切标记,套准标记,颜色条,页面信息, 裁剪标记Crop Mark, 就是给印刷厂裁剪纸张用的.
%%%%%%%%%%%%%%%%%%%%%%%%%%%%%%%%%%%%%%%%% 为打印准备图稿时,打印设备需要几种标记来精确套准图稿元素并校验正确的颜色.
%%%%%%%%%%%%%%%%%%%%%%%%%%%%%%%%%%%%%%%%% http://blog.sina.com.cn/s/blog_5e16f177010172py.html
%%%%
%%%%%%%%%%%%%%%%%%%% setting-011
%%%%
\usepackage{indentfirst} %%  段落缩进
\setlength{\parindent}{2em} %% 段落缩进,2em表示两个汉字
%%%%
%%%%%%%%%%%%%%%%%%%% setting-012
%%%%
%%%% %%%% %%%% \usepackage{lineno} %% 加入行号
%%%% %%%% %%%% %%%% \linenumbers %% 在开始需要标号的位置加入行号,这个是针对单行的,并且公式是不会标记行号的,还有\nolinenumber为结束行号
%%
%% \usepackage[switch]{lineno}
%% \linenumbers %% 这个是针对杂志的投稿要求,需要是双栏,双栏文本使行号位于两侧,即页变位置
%%%%
%%%%%%%%%%%%%%%%%%%% setting-013
%%%%
\usepackage{pdfpages}
%% Inserting a PDF file in Latex
%% https://stackoverflow.com/questions/2739159/inserting-a-pdf-file-in-latex
%%  %% ex-1
%% \includegraphics[scale=0.75,page=2]{multipage.pdf}
%%  %% ex-2
%% \begin{figure}[htp] \centering{
%% \includegraphics[scale=0.75]{sampaper.pdf}}
%% \caption{Experiment 2}
%% \end{figure}
%% %% ex-3
%% \includepdf[pages={1,3,5}]{myfile.pdf}
%% %% ex-4
%% \includepdf[pages={11-12}]{myfile.pdf}
%%%%
%%%%%%%%%%%%%%%%%%%% setting-013+1
%%%%
\usepackage{enumerate}
%%%%
%%%%%%%%%%%%%%%%%%%% setting-013+1
%%%%
\usepackage{multirow}  %% 多行合并的宏包
\usepackage{makecell}  %% thead{对齐方式}{列名称} %% 专门处理表头
\usepackage{threeparttable} %% 给表格添加注释
%% \usepackage{tablefootnote} %% 给表格添加注释
%%%%%%%%%%%%%%%%%%%%
%%%% oiint
\newcommand{\oiint}{\,\bigcirc\kern -17.3pt\int\kern -7.5pt\int}
%%%%
%%%%%%%%%%%%%%%%%%%% setting-013+1
\title{高等数学1-2
  %\thanks{Ex001 - Ex177}
}
%%%% setting-014 五个作者,一个作者对应两个工作单位
%% from: https://blog.csdn.net/robert_chen1988/article/details/79187224
%\author[ab]{Lihui Gao \thanks{lhgao1@gzu.edu.cn}} %%使用\thanks{}定义通讯作者
\author{}
%%%%
%%%%%%%%%%%%%%%%%%%% setting-015
%%%%
\date{\today} %省略的时候自动使用当前日期
% \date{} %如果{}为空不显示日期
%%%%
%%%%%%%%%%%%%%%%%%%% setting-016 设定参考文献的格式
%%%%
\bibliographystyle{plain}
%%%%
%%%%%%%%%%%%%%%%%%%% setting-017 正文开始
\begin{document}
%your content
%%%%%%%%%%%%%%%%%%% setting-018 内容目录
\maketitle
%% \section{}
% \noindent
%% \newpage
%%%%%%%%%%%%%%%%%%
\par\vspace{1em}
%\begin{abstract}
%练习册
%\end{abstract}
%%%%%%%%%%
%% \tableofcontents
%% 
%\section{上册}

%\section{下册练习}

%%%%%%%%%%%%%%%%
%% \newpage
%%     \begin{center}
%%         \quad \\
%%         \quad \\
%%         \fontsize{45}{17} 毕\quad 业\quad 论\quad 文
%%         \vskip 3.5cm
%%         \large 在此打印论文题目,二号黑体  
%%     \end{center}
%%     \vskip 3.5cm
%% 
%%     \begin{quotation}
%%         \fontsize{15}{15}
%%         \par\setlength\parindent{12em}
%%         \quad 
%% 
%%         学\hspace{0.61cm} 院:\underline{共产主义学院\quad}
%% 
%%         专\hspace{0.61cm} 业:\underline{计算机科学技术}
%% 
%%         学生姓名:\underline{\qquad 德彪西\qquad }
%% 
%%         学\hspace{0.61cm} 号:\underline{\quad 00000000000\quad}
%% 
%%         指导教师:\underline{\qquad 傅里叶 \qquad}
%%         \vskip 2cm
%%         \centering
%%         2019年**月**日
%%     \end{quotation}
%%%%%%%%%%%%%%%%%%%%%%%%%%%%%
\begin{titlepage}
        
        \vspace*{34pt}
        %\vspace*{64pt}
        \begin{center}
            \fontsize{44pt}{0} 《\quad 高\quad 等\quad 数\quad 学\quad 》\\
            %\fontsize{48pt}{0} 贵州大学\\
            \vspace*{36pt}\par
            %\fontsize{36pt}{0}{课\quad 程\quad 报\quad 告}\\
            \fontsize{36pt}{0}{练\quad 习\quad 册}\\
            \vspace*{48pt}
            \LARGE(\quad2020\quad~$\thicksim$~\quad 2021\quad~学年度\quad \quad 第\quad 2\quad 学期\quad)\\
            %\fontsize{24pt}{0}(2020~$\thicksim$~2021~学年度\quad 第\,\,2\,\,学期)\\
            \vspace*{48pt}

            \LARGE 课\,\,程\,\,名\,\,称\ \ \underline{\makebox[200pt]{高\,\,等\,\,数\,\,学\,\,1\,\,-\,\,2}}\\
            \vspace*{72pt}

            %\Large 姓名\ \ \underline{\makebox[168pt]{kurrr}}\\\vskip 0.6cm
            \Large 姓\,\,名\ \ \underline{\makebox[168pt]{ }}\\\vskip 1.5cm
            \Large 学\,\,号\ \ \underline{\makebox[168pt]{}}\\\vskip 1.5cm
            %\Large 学号\ \ \underline{\makebox[168pt]{2019XXXXXXX}}\\\vskip 0.6cm
            \Large 学\,\,院\ \ \underline{\makebox[168pt]{}}\\\vskip 1.5cm
            %\Large 学院\ \ \underline{\makebox[168pt]{XXXXXXXXX}}\\\vskip 0.6cm
            \Large 教\,\,师\ \ \underline{\makebox[168pt]{ }}\\
            %\Large 教师\ \ \underline{\makebox[168pt]{XXX}}\\
        \end{center}
\end{titlepage}
%%%%%%%%%%%%%%%%%%%%%%%%%%%%%%%%%%%%%%%%%%%
\newpage
%\hspace{0.25\linewidth}
\framebox{%
  \begin{minipage}{0.25\linewidth}
\quad \quad \quad\par
\quad \quad \quad\par\quad \quad \quad\par
    \quad \quad \quad\par\quad \quad \quad\par\quad \quad \quad\par
\quad \quad \quad \quad 免冠照
\quad \quad \quad\par
\quad \quad \quad\par\quad \quad \quad\par
\quad \quad \quad\par
\quad \quad \quad\par\quad \quad \quad\par
  \end{minipage}}
%
%%--
\quad
\begin{tabular}{|c|l|}
	\hline
     \quad  & \qquad\qquad\qquad\qquad\qquad\qquad\qquad\qquad\qquad\qquad\quad  \\
	\quad & \quad \\
	\quad & \quad \\
	\quad 梦想\quad\quad& \quad \\
	\quad & \quad \\
	\quad & \quad \\ 
	\hline \hline
	\quad & \quad \\
	\quad & \quad \\
	\quad 困难\quad\quad & \quad \\
	\quad & \quad \\
     \quad & \quad \\
	\hline
\end{tabular}

\vskip 7em
%% \section{练习}
\large
%%%%------------------
%%%%------------------ 练习题 
\setcounter{section}{6}
%\setcounter{section}{1}
%%%%------------------
%%%%------------------ 练习题 

\section{第七章\quad 微分方程}
\subsection{第一节\quad 微分方程的基本概念}
\subsubsection{知识点}
\subsubsection{练习题}
%% \subsubsection{竞赛题}
%% \subsubsection{考研题}
%% %%
% \paragraph{a}
% \subparagraph{a}
% \paragraph{a}
% \subparagraph{a}


%% 

\vskip 0.01\textheight
\par\noindent \textbf{001} \quad 已知曲线上点$P(x,y)$处的法线与$x$轴交点为$Q$, 
且线段$PQ$被$y$轴平分, 求该曲线所满足的微分方程。 
\par\noindent \textbf{解答}

%%%%------------------ 练习题 
%%%%------------------

%%%%------------------
%%%%------------------ 练习题 
\vskip 0.34271\textheight

\par\noindent \textbf{003} \quad 函数$y=3\,e\,^{2\,x}$是微分方程$y\,''-4y=0$的\hfil (\quad\quad\quad)%什么解?
\par\noindent \textbf{A} \quad  通解
\par\noindent \textbf{B} \quad  特解



%%%%------------------ 练习题 
%%%%------------------


%%%%------------------
%%%%------------------ 练习题 
\vskip 0.44271\textheight

\newpage
\par\noindent \textbf{004} \quad 微分方程
$$
xy\,'''+2y\,''+x^2y=0
$$
的阶数为\hfill (\quad\quad\quad)
\par\noindent \textbf{A} \quad  1
\par\noindent \textbf{B} \quad  2
\par\noindent \textbf{C} \quad  3
\par\noindent \textbf{D} \quad  4



%%%%------------------ 练习题 
%%%%------------------


%%%%------------------
%%%%------------------ 练习题 
\vskip 0.44271\textheight


\par\noindent \textbf{005} \quad 微分方程
$$L\frac{d\,^2Q}{d\,t\,^2}+R\frac{d\,Q}{d\,t}+\frac{Q}{c}=0$$
的阶数为\hfill (\quad\quad\quad)
\par\noindent \textbf{A} \quad  1
\par\noindent \textbf{B} \quad  2
\par\noindent \textbf{C} \quad  3
\par\noindent \textbf{D} \quad  4



%%%%------------------ 练习题 
%%%%------------------


%%%%------------------
%%%%------------------ 练习题 
\vskip 0.44271\textheight


\par\noindent \textbf{006} \quad 微分方程
$$\frac{d\,\rho}{d\,\theta}+\rho=\sin^2\theta$$
的阶数为\hfill (\quad\quad\quad)
\par\noindent \textbf{A} \quad  1
\par\noindent \textbf{B} \quad  2
\par\noindent \textbf{C} \quad  3
\par\noindent \textbf{D} \quad  4



%%%%------------------ 练习题 
%%%%------------------


%%%%------------------
%%%%------------------ 练习题 
\vskip 0.44271\textheight


\par\noindent \textbf{007} \quad 一个二阶微分方程的通解应含有多少个任意常数\hfill (\quad\quad\quad)
\par\noindent \textbf{A} \quad  1
\par\noindent \textbf{B} \quad  2
\par\noindent \textbf{C} \quad  3
\par\noindent \textbf{D} \quad  4



%%%%------------------ 练习题 
%%%%------------------


%%%%------------------
%%%%------------------ 练习题 
\vskip 0.44271\textheight


\par\noindent \textbf{008} \quad 已知$y=c\,_1\sin (x-c\,_2)$, $y\,|_{x\,=\,\pi}=1$, $y\,'|_{x\,=\,\pi}=0$, 求$c\,_1, c\,_2$
\par\noindent \textbf{ 解答}



%%%%------------------ 练习题 
%%%%------------------


%%%%------------------
%%%%------------------ 练习题 
\vskip 0.44271\textheight


\par\noindent \textbf{009} \quad 求函数$y=ae\,^x-be\,^{-x}+x-1$所满足的微分方程, 其中$a$, $b$为任意常数。
\par\noindent \textbf{ 解答}



%%%%------------------ 练习题 
%%%%------------------


%%%%------------------
%%%%------------------ 练习题 
% \vskip 0.44271\textheight
% 
\newpage
% \par\noindent \textbf{010} \quad 设曲线上点$P(x,y)$处的法线与$x$轴的交点为$Q$, 
% 且线段$PQ$被$y$轴平分, 
% 求该曲线所满足的微分方程。(重复了, 修改)
% \par\noindent \textbf{ 解答}



%%%%------------------ 练习题 
%%%%------------------

\newpage
\subsection{第二节\quad 可分离变量微分方程}
\subsubsection{知识点}
\subsubsection{练习题}
%%%%------------------
%%%%------------------ 练习题 
%\vskip 0.44271\textheight


\par\noindent \textbf{011} \quad 求微分方程
$$\frac{d\,y}{d\,x}+\cos \frac{x-y}{2}=\cos \frac{x+y}{2}$$
的通解。
\par\noindent \textbf{ 解答}



%%%%------------------ 练习题 
%%%%------------------

\newpage
\subsection{第三节\quad 齐次方程}
\subsubsection{知识点}
\subsubsection{练习题}
%%%%------------------
%%%%------------------ 练习题 
%\vskip 0.44271\textheight


\par\noindent \textbf{012} \quad 将方程
$$\int_0^x\left[\,2y\,(t)+\sqrt{t\,^2+y\,^2(t)}\,\right] \quad dt=x\,y\,(x)$$
化为齐将方程的形式。
\par\noindent \textbf{ 解答}



%%%%------------------ 练习题 
%%%%------------------


%%%%------------------
%%%%------------------ 练习题 
\vskip 0.44271\textheight


\par\noindent \textbf{013} \quad 求方程
$$\left(\,x\,^2+y\,^2\,\right)d\,x-x\,y\,d\,y=0$$
的通解。
\par\noindent \textbf{ 解答}



%%%%------------------ 练习题 
%%%%------------------


%%%%------------------
%%%%------------------ 练习题 
\vskip 0.44271\textheight


\par\noindent \textbf{014} \quad 求齐次方程
$$\left(\,y\,^2-3\,x\,^2\,\right)d\,y+2\,x\,y\,d\,x=0$$
满足初始条件$y\,|_{x\,=\,0}=1$
的特解。
\par\noindent \textbf{ 解答}



%%%%------------------ 练习题 
%%%%------------------


%%%%------------------
%%%%------------------ 练习题 
\vskip 0.44271\textheight


\par\noindent \textbf{015} \quad 求方程
$$y\,' = \frac{x+y+1}{x-y-3}$$
的通解。(D7-3齐次方程 (可化为齐次方程的方程))
\par\noindent \textbf{ 解答}



%%%%------------------ 练习题 
%%%%------------------


%%%%------------------
%%%%------------------ 练习题 
\vskip 0.44271\textheight

\newpage
\subsection{第四节\quad 一阶线性微分方程}
\subsubsection{知识点}
\subsubsection{练习题}
\par\noindent \textbf{016} \quad 求方程
$$y\,^3\,d\,x+(\,2\,x\,y\,^2-1)\,d\,y = 0$$
的通解。
\par\noindent \textbf{ 解答}



%%%%------------------ 练习题 
%%%%------------------


%%%%------------------
%%%%------------------ 练习题 
\vskip 0.44271\textheight


\par\noindent \textbf{017} \quad 求微分方程
$$y\,'= \frac{1}{2}\tan ^2\,(\,x+2\,y\,)$$
的通解。
\par\noindent \textbf{ 解答}



%%%%------------------ 练习题 
%%%%------------------


%%%%------------------
%%%%------------------ 练习题 
\vskip 0.44271\textheight


\par\noindent \textbf{018} \quad 求微分方程
$$\frac{dy}{dx}+\cos \frac{x-y}{2}=\cos \frac{x+y}{2}$$
的通解。(和差化积公式)
\par\noindent \textbf{ 解答}



%%%%------------------ 练习题 
%%%%------------------


%%%%------------------
%%%%------------------ 练习题 
\vskip 0.44271\textheight


\newpage
\subsection{第五节\quad 可降阶高阶微分方程}
\subsubsection{知识点}
\subsubsection{练习题}
\par\noindent \textbf{019} \quad 求微分方程
$$y\, '' = x+\sin x$$
的通解。
\par\noindent \textbf{ 解答}



%%%%------------------ 练习题 
%%%%------------------


%%%%------------------
%%%%------------------ 练习题 
\vskip 0.44271\textheight


\par\noindent \textbf{020} \quad 求微分方程
$$y\, '' = y\,' + x$$
的通解。
\par\noindent \textbf{ 解答}



%%%%------------------ 练习题 
%%%%------------------


%%%%------------------
%%%%------------------ 练习题 
\vskip 0.44271\textheight


\par\noindent \textbf{021} \quad 求微分方程
$$y\,^3y\, ''+1 = 0$$
满足初始条件
$y\,|_{x\,=\,1}=1$, $y\,'|_{x\,=\,1}=0$
的特解。
\par\noindent \textbf{ 解答}



%%%%------------------ 练习题 
%%%%------------------


%%%%------------------
%%%%------------------ 练习题 
\vskip 0.44271\textheight


\par\noindent \textbf{022} \quad 求微分方程
$$y\,'' = 3\,\sqrt{y}$$
满足初始条件
$y\,|_{x\,=\,0}=1$, $y\,'|_{x\,=\,0}=2$
的特解。
\par\noindent \textbf{ 解答}



%%%%------------------ 练习题 
%%%%------------------


%%%%------------------
%%%%------------------ 练习题 
\vskip 0.44271\textheight


\par\noindent \textbf{023} \quad 求微分方程
$$y\,'' = e\,^{2x}-\cos x$$
满足初始条件
$y\,|_{x\,=\,0}=0$, $y\,'|_{x\,=\,0}=1$
的特解。
\par\noindent \textbf{ 解答}



%%%%------------------ 练习题 
%%%%------------------


%%%%------------------
%%%%------------------ 练习题 
\vskip 0.44271\textheight


\par\noindent \textbf{024} \quad 求方程
$$\left(1+x\,^2\right)\,\frac{d\,^2y}{d\,x\,^2}-2\,x\,\frac{d\,y}{d\,x}=0$$
的通解。
\par\noindent \textbf{ 解答}



%%%%------------------ 练习题 
%%%%------------------


%%%%------------------
%%%%------------------ 练习题 
\vskip 0.44271\textheight


\newpage
\subsection{第六节\quad 高阶线性微分方程}
\subsubsection{知识点}
\subsubsection{练习题}
\par\noindent \textbf{025} \quad 方程
$$\frac{d\,^2y}{d\,x\,^2}+P(x)\frac{d\,y}{d\,x}+Q(x)\,y=0$$
为\hfill (\quad\quad\quad)
\par\noindent \textbf{A} \quad 二阶非齐次线性微分方程
\par\noindent \textbf{B} \quad 二阶齐次线性微分方程



%%%%------------------ 练习题 
%%%%------------------


%%%%------------------
%%%%------------------ 练习题 
\vskip 0.44271\textheight


\par\noindent \textbf{026} \quad 函数
$$y_1(x)=\sin 2x, \quad y_2(x)=6\sin x\cos x$$
\hfill (\quad\quad\quad)
\par\noindent \textbf{A} \quad 线性相关
\par\noindent \textbf{B} \quad 线性无关



%%%%------------------ 练习题 
%%%%------------------


%%%%------------------
%%%%------------------ 练习题 
\vskip 0.44271\textheight


\par\noindent \textbf{027} \quad 设
$y_1(x), \quad y_2(x)$
是方程
$$\frac{d\,^2y}{d\,x\,^2}+P(x)\frac{d\,y}{d\,x}+Q(x)\,y=0$$
的两个线性无关的特解, 则
$$y=C_1y_1(x)+C_2y_2(x)$$
是该方程的
\hfill (\quad\quad\quad)
\par\noindent \textbf{A} \quad 通解
\par\noindent \textbf{B} \quad 特解



%%%%------------------ 练习题 
%%%%------------------


%%%%------------------
%%%%------------------ 练习题 
\vskip 0.44271\textheight


\par\noindent \textbf{028} \quad 
$y=C_1\cos x+C_2\sin x$
是方程
$$y\,''+y=0$$
的
\hfill (\quad\quad\quad)
\par\noindent \textbf{A} \quad 通解
\par\noindent \textbf{B} \quad 特解



%%%%------------------ 练习题 
%%%%------------------


%%%%------------------
%%%%------------------ 练习题 
\vskip 0.44271\textheight


\par\noindent \textbf{029} \quad 
$y=C_1\cos x+C_2\sin x+x^2-2$
是方程
$$y\,''+y=x^2$$
的
\hfill (\quad\quad\quad)
\par\noindent \textbf{A} \quad 通解
\par\noindent \textbf{B} \quad 特解



%%%%------------------ 练习题 
%%%%------------------


%%%%------------------
%%%%------------------ 练习题 
\vskip 0.44271\textheight


\par\noindent \textbf{030} \quad 设$y_1^*$与$y_2^*$分别是方程
$$y\,''+P(x)y\,'+Q(x)y=f\,_1(x)$$
与
$$y\,''+P(x)y\,'+Q(x)y=f\,_2(x)$$
的特解, 则$y_1^*+y_2^*$是方程
$$y\,''+P(x)y\,'+Q(x)y=f\,_1(x)+f\,_2(x)$$
的
\hfill (\quad\quad\quad)
\par\noindent \textbf{A} \quad 通解
\par\noindent \textbf{B} \quad 特解



%%%%------------------ 练习题 
%%%%------------------


%%%%------------------
%%%%------------------ 练习题 
\vskip 0.44271\textheight


\par\noindent \textbf{031} \quad 求方程
$$\frac{d\,^2y}{d\,x\,^2}-\frac{1}{x}\,\frac{d\,y}{d\,x}=x$$
的通解。
\par\noindent \textbf{ 解答}



%%%%------------------ 练习题 
%%%%------------------


%%%%------------------
%%%%------------------ 练习题 
\vskip 0.44271\textheight


\par\noindent \textbf{032} \quad 求方程
$$y\,''+\frac{x}{1-x}\,y\,'-\frac{1}{1-x}\,y=x-1$$
的通解。
\par\noindent \textbf{ 解答}



%%%%------------------ 练习题 
%%%%------------------


%%%%------------------
%%%%------------------ 练习题 
\vskip 0.44271\textheight


\newpage
\subsection{第七节\quad 常系数齐次线性微分方程}
\subsubsection{知识点}
\subsubsection{练习题}
\par\noindent \textbf{033} \quad 设$y_1(x)=e\,^x$为齐次方程
$$y\,''-2\,y\,'+y=0$$
的解, 求非齐次方程
$$y\,''-2\,y\,'+y=\frac{1}{x}\,e\,^x$$
的通解。
\par\noindent \textbf{ 解答}



%%%%------------------ 练习题 
%%%%------------------


%%%%------------------
%%%%------------------ 练习题 
\vskip 0.44271\textheight


\par\noindent \textbf{034} \quad 求微分方程
$$y\,''-2\,y\,'-3y=0$$
的通解。
\par\noindent \textbf{ 解答}



%%%%------------------ 练习题 
%%%%------------------


%%%%------------------
%%%%------------------ 练习题 
\vskip 0.44271\textheight


\newpage
\subsection{第八节\quad 常系数非齐次线性微分方程}
\subsubsection{知识点}
\subsubsection{练习题}
\par\noindent \textbf{035} \quad 写出二阶常系数非齐次线性微分方程的一般形式。
\par\noindent \textbf{ 解答}



%%%%------------------ 练习题 
%%%%------------------


%%%%------------------
%%%%------------------ 练习题 
\vskip 0.44271\textheight


\par\noindent \textbf{036} \quad 写出方程
$$y\,''+5\,y\,'+6\,y=3\,x\,e^{-2\,x}$$
的特解形式。
\par\noindent \textbf{ 解答}



%%%%------------------ 练习题 
%%%%------------------


%%%%------------------
%%%%------------------ 练习题 
\vskip 0.44271\textheight


\par\noindent \textbf{037} \quad 求方程
$$y\,''-2\,y\,'-3\,y=3\,x+1$$
的一个特解。
\par\noindent \textbf{ 解答}



%%%%------------------ 练习题 
%%%%------------------


%%%%------------------
%%%%------------------ 练习题 
\vskip 0.44271\textheight


\par\noindent \textbf{038} \quad 求方程
$$y\,''-3\,y\,'+2\,y=x\,e\,^{2\,x}$$
的一个特解。
\par\noindent \textbf{ 解答}



%%%%------------------ 练习题 
%%%%------------------


%%%%------------------
%%%%------------------ 练习题 
\vskip 0.44271\textheight


\par\noindent \textbf{039} \quad 求方程
$$y\,''-3\,y\,'+2\,y=x\,e\,^{2\,x}$$
的通解。
\par\noindent \textbf{ 解答}



%%%%------------------ 练习题 
%%%%------------------


%%%%------------------
%%%%------------------ 练习题 
\vskip 0.44271\textheight


\par\noindent \textbf{040} \quad 求方程
$$y\,''+y=x+e\,^x$$
的一个特解。
\par\noindent \textbf{ 解答}



%%%%------------------ 练习题 
%%%%------------------

%% Ex001 - Ex040



%%%%------------------
%%%%------------------ 练习题 
\vskip 0.44271\textheight


\par\noindent \textbf{041} \quad 求方程
$$y\,''+y=x+e\,^x$$
的通解。
\par\noindent \textbf{ 解答}



%%%%------------------ 练习题 
%%%%------------------


%%%%------------------
%%%%------------------ 练习题 
\vskip 0.44271\textheight


\par\noindent \textbf{042} \quad 求方程
$$y\,'''+3\,y\,''+3\,y\,'+y=e\,^x$$
的通解。
\par\noindent \textbf{ 解答}



%%%%------------------ 练习题 
%%%%------------------


%%%%------------------
%%%%------------------ 练习题 
\vskip 0.44271\textheight


\par\noindent \textbf{043} \quad 求方程
$$y\,''+y=4\sin x$$
的通解。
\par\noindent \textbf{ 解答}



%%%%------------------ 练习题 
%%%%------------------


%%%%------------------
%%%%------------------ 练习题 
\vskip 0.44271\textheight


\par\noindent \textbf{044} \quad 求方程
$$y\,''+y=x\cos 2x$$
对应齐次方程的通解。
\par\noindent \textbf{ 解答}



%%%%------------------ 练习题 
%%%%------------------


%%%%------------------
%%%%------------------ 练习题 
\vskip 0.44271\textheight


\par\noindent \textbf{045} \quad 求方程
$$y\,''+y=x\cos 2x$$
的通解。
\par\noindent \textbf{ 解答}



%%%%------------------ 练习题 
%%%%------------------


%%%%------------------
%%%%------------------ 练习题 
\vskip 0.44271\textheight


\par\noindent \textbf{046} \quad 求方程
$$y\,''+y\,'=2\,x\,^2 e\,^x$$
的通解。
\par\noindent \textbf{ 解答}



%%%%------------------ 练习题 
%%%%------------------


%%%%------------------
%%%%------------------ 练习题 
\vskip 0.44271\textheight


\par\noindent \textbf{047} \quad 求方程
$$y\,''+2y\,'+5y=\sin 2x$$
的通解。
\par\noindent \textbf{ 解答}



%%%%------------------ 练习题 
%%%%------------------


%%%%------------------
%%%%------------------ 练习题 
\vskip 0.44271\textheight


\newpage
\section{第八章\quad 空间解析几何与向量代数}
\subsection{第一节\quad 向量及运算}
\subsubsection{知识点}
\subsubsection{练习题}
\par\noindent \textbf{049} \quad 化简
$$\vec{a}-\vec{b}+5\left(-\frac{1}{2}\vec{b}+\frac{\vec{b}-3\vec{a}}{5}\right)$$
\par\noindent \textbf{ 解答}



%%%%------------------ 练习题 
%%%%------------------


%%%%------------------
%%%%------------------ 练习题 
\vskip 0.44271\textheight


\newpage
\subsection{第二节\quad 点积叉积}
\subsubsection{知识点}
\subsubsection{练习题}
\par\noindent \textbf{050} \quad 名词解释: 数量积、 向量积。
\par\noindent \textbf{ 解答}



%%%%------------------ 练习题 
%%%%------------------


%%%%------------------
%%%%------------------ 练习题 
\vskip 0.44271\textheight


\par\noindent \textbf{051} \quad 已知三点$M\,(1,1,1)$, $A\,(2,2,1)$, $B\,(2,1,2)$, 求$\angle AMB$
\par\noindent \textbf{ 解答}



%%%%------------------ 练习题 
%%%%------------------


%%%%------------------
%%%%------------------ 练习题 
\vskip 0.44271\textheight


\par\noindent \textbf{052} \quad 请完整叙述三个非零向量共面的充要条件。
\par\noindent \textbf{ 解答}



%%%%------------------ 练习题 
%%%%------------------


%%%%------------------
%%%%------------------ 练习题 
\vskip 0.44271\textheight


\par\noindent \textbf{053} \quad 设$\vec{a}$, $\vec{b}$的夹角$\displaystyle\frac{3}{4}\pi$, 
$|\vec{a}|=\sqrt{2}$, $|\vec{b}|=3$,
求$\left|\,\vec{a}-\vec{b}\,\right|$
\par\noindent \textbf{ 解答}



%%%%------------------ 练习题 
%%%%------------------


%%%%------------------
%%%%------------------ 练习题 
\vskip 0.44271\textheight


\newpage
\subsection{第三节\quad 曲面方程}
\subsubsection{知识点}
\subsubsection{练习题}
\par\noindent \textbf{054} \quad 名词解释: 椭球面、 椭圆柱面。
\par\noindent \textbf{ 解答}



%%%%------------------ 练习题 
%%%%------------------


%%%%------------------
%%%%------------------ 练习题 
\vskip 0.44271\textheight


\par\noindent \textbf{055} \quad 名词解释: 双曲抛物面、 椭圆抛物面。
\par\noindent \textbf{ 解答}



%%%%------------------ 练习题 
%%%%------------------


%%%%------------------
%%%%------------------ 练习题 
\vskip 0.44271\textheight


\par\noindent \textbf{056} \quad 名词解释: 单叶双曲面、 双叶双曲面。
\par\noindent \textbf{ 解答}



%%%%------------------ 练习题 
%%%%------------------


%%%%------------------
%%%%------------------ 练习题 
\vskip 0.44271\textheight


\par\noindent \textbf{057} \quad 指出下列方程
$$x\,^2+y\,^2=9$$
在空间解析几何中的图形。
\par\noindent \textbf{ 解答}



%%%%------------------ 练习题 
%%%%------------------


%%%%------------------
%%%%------------------ 练习题 
\vskip 0.44271\textheight


\newpage
\subsection{第四节\quad 空间曲线}
\subsubsection{知识点}
\subsubsection{练习题}
\par\noindent \textbf{058} \quad 写出曲线$C$\,:
\begin{equation*}    %f(x) =
 \begin{cases}
    x\,^2+y\,^2+z\,^2 & =1 \\
    x\,^2+(y-1)\,^2+(z-1)\,^2 & =1
 \end{cases}                
\end{equation*}
在$xoy$面上的投影曲线方程。
\par\noindent \textbf{ 解答}



%%%%------------------ 练习题 
%%%%------------------


%%%%------------------
%%%%------------------ 练习题 
\vskip 0.44271\textheight


\newpage
\subsection{第五节\quad 平面方程}
\subsubsection{知识点}
\subsubsection{练习题}
\par\noindent \textbf{059} \quad 设一平面通过已知点$M_0(x_0,y_0,z_0)$, 且垂直于非零向量$\vec{n}=(A,B,C)$,
求该平面方程。
\par\noindent \textbf{ 解答}



%%%%------------------ 练习题 
%%%%------------------


%%%%------------------
%%%%------------------ 练习题 
\vskip 0.44271\textheight


\par\noindent \textbf{060} \quad 计算
$$
\left |\begin{array}{cccc}
x-2 & y+1  & z-4 \\
-3  & 4    & -6  \\
-2  & 3    & -1 \\
\end{array}\right|
$$
\par\noindent \textbf{ 解答}



%%%%------------------ 练习题 
%%%%------------------


%%%%------------------
%%%%------------------ 练习题 
\vskip 0.44271\textheight


\par\noindent \textbf{061} \quad 名词解释: 截距式方程、 三点式方程。
\par\noindent \textbf{ 解答}



%%%%------------------ 练习题 
%%%%------------------


%%%%------------------
%%%%------------------ 练习题 
\vskip 0.44271\textheight


\par\noindent \textbf{062} \quad 求通过$x$轴和点$(4,-3,-1)$的平面方程。
\par\noindent \textbf{ 解答}



%%%%------------------ 练习题 
%%%%------------------


%%%%------------------
%%%%------------------ 练习题 
\vskip 0.44271\textheight

\newpage
\par\noindent \textbf{063} \quad 请完整叙述两平面的位置关系。
\par\noindent \textbf{ 解答}



%%%%------------------ 练习题 
%%%%------------------


%%%%------------------
%%%%------------------ 练习题 
\vskip 0.44271\textheight


\par\noindent \textbf{064} \quad 请过点$(1,1,1)$且垂直于两平面$x-y+z=7$和$3x+2y-12z+5=0$的平面方程。
\par\noindent \textbf{ 解答}



%%%%------------------ 练习题 
%%%%------------------


%%%%------------------
%%%%------------------ 练习题 
\vskip 0.44271\textheight


\newpage
\subsection{第六节\quad 空间直线}
\subsubsection{知识点}
\subsubsection{练习题}
\par\noindent \textbf{065} \quad 名词解释: 直线的对称式方程。
\par\noindent \textbf{ 解答}



%%%%------------------ 练习题 
%%%%------------------


%%%%------------------
%%%%------------------ 练习题 
\vskip 0.44271\textheight


\par\noindent \textbf{066} \quad 写出直线
\begin{equation*}    %f(x) =
 \begin{cases}
    x+y+z+1 & =0 \\
    2x-y+3z+4 & =0
 \end{cases}                
\end{equation*}
的对称式方程和参数式方程。
\par\noindent \textbf{ 解答}



%%%%------------------ 练习题 
%%%%------------------


%%%%------------------
%%%%------------------ 练习题 
\vskip 0.44271\textheight


\par\noindent \textbf{067} \quad 一直线过点$A(1,2,1)$且垂直于直线
$$ L_1: \frac{x-1}{3}=\frac{y}{2}=\frac{z+1}{1}$$
又和直线
$$ L_2: \frac{x}{2}=y=\frac{z}{-1}$$
相交, 求此直线方程。
\par\noindent \textbf{ 解答}



%%%%------------------ 练习题 
%%%%------------------


%%%%------------------
%%%%------------------ 练习题 
\vskip 0.44271\textheight


\par\noindent \textbf{068} \quad 求与两平面$x-4z=3$和$2x-y-5z=1$的交线平行, 且过点$(-3,2,5)$的直线方程。
\par\noindent \textbf{ 解答}



%%%%------------------ 练习题 
%%%%------------------


%%%%------------------
%%%%------------------ 练习题 
\vskip 0.44271\textheight


\par\noindent \textbf{069} \quad 求直线
$$\frac{x-2}{1}=\frac{y-3}{1}=\frac{z-4}{2}$$
与平面
$$2x+y+z-6=0$$
的交点。
\par\noindent \textbf{ 解答}



%%%%------------------ 练习题 
%%%%------------------


%%%%------------------
%%%%------------------ 练习题 
\vskip 0.44271\textheight


\par\noindent \textbf{070} \quad 求过点$(2,1,3)$且与直线
$$\frac{x+1}{3}=\frac{y-1}{2}=\frac{z}{-1}$$
垂直相交的直线方程。
\par\noindent \textbf{ 解答}



%%%%------------------ 练习题 
%%%%------------------


%%%%------------------
%%%%------------------ 练习题 
\vskip 0.44271\textheight


\par\noindent \textbf{071} \quad 求直线
\begin{equation*}    %f(x) =
 \begin{cases}
    x+y-z-1 & =0 \\
    x-y+z+1 & =0
 \end{cases}                
\end{equation*}
在平面
$$x+y+z=0$$
上的投影直线方程。
\par\noindent \textbf{ 解答}



%%%%------------------ 练习题 
%%%%------------------


%%%%------------------
%%%%------------------ 练习题 
\vskip 0.44271\textheight


\newpage
\section{第九章\quad 多元函数微分法及其应用}
\subsection{第一节\quad 基本概念}
\subsubsection{知识点}
\subsubsection{练习题}
\par\noindent \textbf{072} \quad 名词解释: 内点、 外点、 边界点。
\par\noindent \textbf{ 解答}



%%%%------------------ 练习题 
%%%%------------------


%%%%------------------
%%%%------------------ 练习题 
\vskip 0.44271\textheight


\par\noindent \textbf{073} \quad 设
$$f\,(x,y)=\left(x\,^2+y\,^2\right)\,\sin \frac{1}{x\,^2+y\,^2}$$
其中$x\,^2+y\,^2\neq 0$, 证明
$$\lim_{x\to 0 \atop y\to 0} f\,(x,y) ​=0$$
\par\noindent \textbf{ 解答}



%%%%------------------ 练习题 
%%%%------------------


%%%%------------------
%%%%------------------ 练习题 
\vskip 0.44271\textheight


\par\noindent \textbf{074} \quad 
$$f\,(x,y)=\frac{x\,y}{x\,^2+y\,^2}$$
在点$(0,0)$处的极限\hfill (\quad\quad\quad)
\par\noindent \textbf{A} \quad 存在
\par\noindent \textbf{B} \quad 不存在



%%%%------------------ 练习题 
%%%%------------------


%%%%------------------
%%%%------------------ 练习题 
\vskip 0.44271\textheight


\par\noindent \textbf{075} \quad 
$$\lim_{x\to 0 \atop y\to 0} \frac{1-\cos (x^2+y^2)}{(x^2+y^2)x^2y^2} ​
= \quad\quad\quad\quad\quad\quad (\quad\quad\quad)$$
\par\noindent \textbf{A} \quad $0$
\par\noindent \textbf{B} \quad $1$
\par\noindent \textbf{C} \quad $2$
\par\noindent \textbf{D} \quad $\infty$



%%%%------------------ 练习题 
%%%%------------------


%%%%------------------
%%%%------------------ 练习题 
\vskip 0.44271\textheight


\par\noindent \textbf{076} \quad 名词解释: 二重极限、 累次极限。
\par\noindent \textbf{ 解答}



%%%%------------------ 练习题 
%%%%------------------


%%%%------------------
%%%%------------------ 练习题 
\vskip 0.44271\textheight


\par\noindent \textbf{077} \quad 函数
\begin{equation*}    f(x,y) =
 \begin{cases}
    \displaystyle \frac{xy}{x\,^2+y\,^2}, & x\,^2+y\,^2\neq 0 \\
    0,                  & x\,^2+y\,^2=0
 \end{cases}                
\end{equation*}
在点$(0,0)$处的极限\hfill (\quad\quad\quad)
\par\noindent \textbf{A} \quad 存在
\par\noindent \textbf{B} \quad 不存在



%%%%------------------ 练习题 
%%%%------------------


%%%%------------------
%%%%------------------ 练习题 
\vskip 0.44271\textheight


\par\noindent \textbf{078} \quad 计算
$$\lim_{x\to 0 \atop y\to 0} \frac{\sqrt{x\,y+1}-1}{x\,y}$$



%%%%------------------ 练习题 
%%%%------------------


%%%%------------------
%%%%------------------ 练习题 
\vskip 0.44271\textheight


\par\noindent \textbf{079} \quad 设
$$f\,\left(\,x\,y, \frac{y\,^2}{x}\right)=x\,^2+y\,^2$$ 
求
$$f\,\left(\frac{y\,^2}{x},x\,y\right)$$



%%%%------------------ 练习题 
%%%%------------------


%%%%------------------
%%%%------------------ 练习题 
\vskip 0.44271\textheight


\par\noindent \textbf{080} \quad 计算
$$\lim_{x\to 0 \atop y\to 0} \frac{x\,\ln(1+x\,y)}{x+y}$$
\par\noindent \textbf{A} \quad 存在
\par\noindent \textbf{B} \quad 不存在



%%%%------------------ 练习题 
%%%%------------------

%% %% Ex041 - Ex080




%%%%------------------
%%%%------------------ 练习题 
\vskip 0.44271\textheight


\newpage
\subsection{第二节\quad 偏导数}
\subsubsection{知识点}
\subsubsection{练习题}
\par\noindent \textbf{081} \quad 名词解释: 二元函数的偏导数。
\par\noindent \textbf{ 解答}



%%%%------------------ 练习题 
%%%%------------------


%%%%------------------
%%%%------------------ 练习题 
\vskip 0.44271\textheight


\par\noindent \textbf{082} \quad 名词解释: 三元函数的偏导数。
\par\noindent \textbf{ 解答}



%%%%------------------ 练习题 
%%%%------------------


%%%%------------------
%%%%------------------ 练习题 
\vskip 0.44271\textheight

\newpage
\par\noindent \textbf{083} \quad "若函数$f$在某点各偏导数都存在, 则$f$在该点一定连续。" 该命题是否为真? 若该命题不真, 
请举一个反例。
\par\noindent \textbf{ 解答}



%%%%------------------ 练习题 
%%%%------------------


%%%%------------------
%%%%------------------ 练习题 
\vskip 0.44271\textheight


\par\noindent \textbf{084} \quad 求
$$r=\sqrt{x\,^2+y\,^2+z\,^2}$$
的偏导数。
\par\noindent \textbf{ 解答}



%%%%------------------ 练习题 
%%%%------------------


%%%%------------------
%%%%------------------ 练习题 
\vskip 0.44271\textheight


\newpage
\par\noindent \textbf{第二节\quad 偏导数(续)(高阶偏导数)}
\par\noindent \textbf{085} \quad 写出二元函数混合偏导数的定义。
\par\noindent \textbf{ 解答}



%%%%------------------ 练习题 
%%%%------------------


%%%%------------------
%%%%------------------ 练习题 
\vskip 0.44271\textheight


\par\noindent \textbf{086} \quad 若$f_{xy}(x,y)$和$f_{yx}(x,y)$都在点$(x_0,y_0)$连续, 则
$$f_{xy}(x_0,y_0) = f_{yx}(x_0,y_0)$$
\par\noindent \textbf{A} \quad 正确
\par\noindent \textbf{B} \quad 错误



%%%%------------------ 练习题 
%%%%------------------


%%%%------------------
%%%%------------------ 练习题 
\vskip 0.44271\textheight


\par\noindent \textbf{087} \quad 设$z=f\,(u)$, 方程
$$u=\phi(u)+\int_y^x p(t)dt$$
确定$u$是$x$, $y$的函数, 其中$f\,(u)$, $\phi(u)$可微, $p\,(t)$, $\phi'(u)$连续, 
且$\phi'(u)\neq 1$, 求
$$p\,(y)\,\frac{\partial z}{\partial x} + p\,(x)\,\frac{\partial z}{\partial y}$$
\par\noindent \textbf{ 解答}



%%%%------------------ 练习题 
%%%%------------------


%%%%------------------
%%%%------------------ 练习题 
\vskip 0.44271\textheight


\newpage
\subsection{第三节\quad 全微分}
\subsubsection{知识点}
\subsubsection{练习题}
\par\noindent \textbf{088} \quad 求
$$z=x\,y+\frac{x}{y}$$
的全微分。
\par\noindent \textbf{ 解答}



%%%%------------------ 练习题 
%%%%------------------


%%%%------------------
%%%%------------------ 练习题 
\vskip 0.44271\textheight


\par\noindent \textbf{088a} \quad 名词解释: 全微分。
\par\noindent \textbf{ 解答}



%%%%------------------ 练习题 
%%%%------------------


%%%%------------------
%%%%------------------ 练习题 
\vskip 0.44271\textheight


\par\noindent \textbf{089} \quad 考虑二元函数$f\,(x,y)$的下面四条性质:\par
1. $f\,(x,y)$在点$(x\,_0,y\,_0)$连续\par
2. $f\,_x(x,y)$, $f\,_y(x,y)$在点$(x\,_0,y\,_0)$连续\par
3. $f\,(x,y)$在点$(x\,_0,y\,_0)$可微\par
4. $f\,_x(x\,_0,y\,_0)$, $f\,_y(x\,_0,y\,_0)$存在\par
若用"P$\Rightarrow$Q"表示可由性质P推出性质Q, 则下列选项正确的是\hfill (\quad\quad\quad)
\par\noindent \textbf{A} \quad 2$\Rightarrow$3$\Rightarrow$1
\par\noindent \textbf{B} \quad 3$\Rightarrow$2$\Rightarrow$1
\par\noindent \textbf{C} \quad 3$\Rightarrow$4$\Rightarrow$1
\par\noindent \textbf{D} \quad 3$\Rightarrow$1$\Rightarrow$4



%%%%------------------ 练习题 
%%%%------------------


%%%%------------------
%%%%------------------ 练习题 
\vskip 0.44271\textheight


\newpage
\subsection{第四节\quad 复合求导}
\subsubsection{知识点}
\subsubsection{练习题}

\par\noindent \textbf{090} \quad 设$z=u\,^2+v\,^2$, 而$u=x+y$, $v=x-y$, 求$\displaystyle\frac{\partial z}{\partial x}$



%%%%------------------ 练习题 
%%%%------------------


%%%%------------------
%%%%------------------ 练习题 
\vskip 0.44271\textheight


\par\noindent \textbf{091} \quad 设$z=u\,^2\ln v$, 而$\displaystyle u=\frac{x}{y}$\,, $v=3\,x-2\,y$, 求$\displaystyle\frac{\partial z}{\partial x}$



%%%%------------------ 练习题 
%%%%------------------


%%%%------------------
%%%%------------------ 练习题 
\vskip 0.44271\textheight


\par\noindent \textbf{092} \quad 设$z=\arctan (x\,y)$, 而$y=e\,^x$, 求$\displaystyle\frac{\partial z}{\partial x}$



%%%%------------------ 练习题 
%%%%------------------


%%%%------------------
%%%%------------------ 练习题 
\vskip 0.44271\textheight


\par\noindent \textbf{093} \quad 求
$$u=f\left(\,x\,^2-y\,^2, e\,^{x\,y}\,\right)$$
的一阶偏导数, 其中$f$具有一阶连续偏导数。



%%%%------------------ 练习题 
%%%%------------------


%%%%------------------
%%%%------------------ 练习题 
\vskip 0.44271\textheight


\par\noindent \textbf{094} \quad 设$z=f\,(x\,^2+y\,^2)$, 其中$f$具有二阶导数, 求
$\displaystyle\frac{\partial \,^2z}{\partial x\,^2}$, 
$\displaystyle\frac{\partial \,^2z}{\partial x\,\partial y}$,
$\displaystyle\frac{\partial \,^2z}{\partial y\,^2}$



%%%%------------------ 练习题 
%%%%------------------


%%%%------------------
%%%%------------------ 练习题 
\vskip 0.44271\textheight


\newpage
\subsection{第五节\quad 隐函数求导}
\subsubsection{知识点}
\subsubsection{练习题}

\par\noindent \textbf{095} \quad 设$\sin y+e\,^x-x\,y\,^2=0$, 求
$\displaystyle\frac{d\,y}{d\,x}$



%%%%------------------ 练习题 
%%%%------------------


%%%%------------------
%%%%------------------ 练习题 
\vskip 0.44271\textheight


\par\noindent \textbf{096} \quad 设$x+2\,y+z-2\,\sqrt{x\,y\,z}=0$, 求
$\displaystyle\frac{\partial\,z}{\partial\,x}$,
$\displaystyle\frac{\partial\,z}{\partial\,y}$



%%%%------------------ 练习题 
%%%%------------------


%%%%------------------
%%%%------------------ 练习题 
\vskip 0.44271\textheight


\par\noindent \textbf{097} \quad 如果函数$f\,(x,y)$满足如下条件中的哪一条, 则该函数在点$(x\,_0,y\,_0)$处连续。
\hfill (\quad\quad\quad)
\par\noindent \textbf{A} \quad $\displaystyle \lim_{x\rightarrow x\,_0}f\,(x,y\,_0)=f\,(x\,_0,y\,_0)$, 
且$\displaystyle\lim_{y\rightarrow y\,_0}f\,(x\,_0,y)=f\,(x\,_0,y\,_0)$
\par\noindent \textbf{B} \quad $f\,(x,y)$在$(x\,_0,y\,_0)$处沿$L$方向有$\displaystyle \frac{\partial \,f}{\partial \,L}$
\par\noindent \textbf{C} \quad $f\,(x,y)$有偏导数$f\,_x(x\,_0,y\,_0)$, $f\,_y(x\,_0,y\,_0)$
\par\noindent \textbf{D} \quad $f\,(x,y)$在$(x\,_0,y\,_0)$处可微分



%%%%------------------ 练习题 
%%%%------------------


%%%%------------------
%%%%------------------ 练习题 
\vskip 0.44271\textheight


\par\noindent \textbf{098} \quad 二元函数
\begin{equation*}    f\,(x,y) =
 \begin{cases}
    \displaystyle\frac{xy}{x^2+y^2}, & (x,y)\neq (0,0) \\
    0,                  & (x,y)=(0,0)
 \end{cases}                
\end{equation*}
在$(0,0)$点处\hfill (\quad\quad\quad)
\par\noindent \textbf{A} \quad 连续、 偏导数存在
\par\noindent \textbf{B} \quad 连续、 偏导数不存在
\par\noindent \textbf{C} \quad 不连续、 偏导数存在
\par\noindent \textbf{D} \quad 不连续、 偏导数不存在



%%%%------------------ 练习题 
%%%%------------------


%%%%------------------
%%%%------------------ 练习题 
\vskip 0.44271\textheight


\par\noindent \textbf{099} \quad 在平面$x+y+z=1$上求一直线, 使它与直线
$$\frac{x-1}{1}=\frac{y}{1}=\frac{z+2}{-1}$$
垂直相交。(调整到“D8-6空间直线”)
\par\noindent \textbf{ 解答}



%%%%------------------ 练习题 
%%%%------------------


%%%%------------------
%%%%------------------ 练习题 
\vskip 0.44271\textheight


\par\noindent \textbf{100} \quad 设$z=z\,(x,y)$由方程$x+y-z=e\,^z$所确定, 求
$\displaystyle\frac{\partial \,^2 z}{\partial \,x\,\partial \,y}$
\par\noindent \textbf{ 解答}



%%%%------------------ 练习题 
%%%%------------------


%%%%------------------
%%%%------------------ 练习题 
\vskip 0.44271\textheight


\newpage
\subsection{第六节\quad 几何中的应用}
\subsubsection{知识点}
\subsubsection{练习题}

\par\noindent \textbf{101} \quad 求曲线$x\,^2+y\,^2+z\,^2=6$, $x+y+z=0$ 在点$M\,(1,-2,1)$处的切线方程与法平面方程。
\par\noindent \textbf{ 解答}



%%%%------------------ 练习题 
%%%%------------------


%%%%------------------
%%%%------------------ 练习题 
\vskip 0.44271\textheight


\par\noindent \textbf{102} \quad 求曲线
\begin{equation*}    %f\,(x,y) =
 \begin{cases}
    x\,^2+y\,^2+z\,^2-3\,x & =0,  \\
    2\,x-3\,y+5\,z-4 & =0
 \end{cases}                
\end{equation*}
在点$(1,1,1)$处的切线方程与法平面方程。
\par\noindent \textbf{ 解答}



%%%%------------------ 练习题 
%%%%------------------


%%%%------------------
%%%%------------------ 练习题 
\vskip 0.44271\textheight


\par\noindent \textbf{103} \quad 如果平面
$$3\,x+\lambda \,y-3\,z+16=0$$
与椭球面
$$3\,x\,^2+y\,^2+z\,^2=16$$
相切, 求$\lambda$
\par\noindent \textbf{ 解答}



%%%%------------------ 练习题 
%%%%------------------


%%%%------------------
%%%%------------------ 练习题 
\vskip 0.44271\textheight


\newpage
\subsection{第七节\quad 方向导数与梯度}
\subsubsection{知识点}
\subsubsection{练习题}
\par\noindent \textbf{104} \quad 函数
$$z=3\,x\,^2y - y\,^2$$
在点$P\,(2,3)$沿曲线
$$y=x\,^2-1$$
朝$x$增大方向的方向导数。
\par\noindent \textbf{ 解答}



%%%%------------------ 练习题 
%%%%------------------


%%%%------------------
%%%%------------------ 练习题 
\vskip 0.44271\textheight


\par\noindent \textbf{105} \quad 函数
$$f\,(x,y,z)=x\,^2 + y\,^z$$
求函数$f$在点$P\,(1,1,1)$沿增加最快方向的方向导数。
\par\noindent \textbf{ 解答}



%%%%------------------ 练习题 
%%%%------------------


%%%%------------------
%%%%------------------ 练习题 
\vskip 0.44271\textheight


\newpage
\subsection{第八节\quad 极值与最值}
\subsubsection{知识点}
\subsubsection{练习题}
\par\noindent \textbf{106} \quad 函数
$f\,(x,y)$
在点$(0,0)$的某个邻域内连续, 且
$$\lim_{x\to 0 \atop y\to 0} \frac{f\,(x,y)-x\,y}{(x\,^2+y\,^2)\,^2} ​=1$$
则\hfill (\quad\quad\quad)
\par\noindent \textbf{A} \quad 点$(0,0)$不是$f\,(x,y)$的极值点
\par\noindent \textbf{B} \quad 点$(0,0)$是$f\,(x,y)$的极大值点
\par\noindent \textbf{C} \quad 点$(0,0)$是$f\,(x,y)$的极小值点
\par\noindent \textbf{D} \quad 根据条件无法判断点$(0,0)$是否为$f\,(x,y)$的极值点
\par\noindent \textbf{ 解答}



%%%%------------------ 练习题 
%%%%------------------


%%%%------------------
%%%%------------------ 练习题 
\vskip 0.44271\textheight


\par\noindent \textbf{107} \quad 求函数
$f\,(x,y)=x\,^3-y\,^3+3\,x\,^2+3\,y\,^2-9\,x$的极值。
\par\noindent \textbf{ 解答}



%%%%------------------ 练习题 
%%%%------------------


%%%%------------------
%%%%------------------ 练习题 
\vskip 0.44271\textheight


\par\noindent \textbf{108} \quad 
点$(0,0)$是函数$z=x\,^3+y\,^3$的极值点。 \hfill (\quad\quad\quad)
\par\noindent \textbf{A} \quad 正确
\par\noindent \textbf{B} \quad 错误



%%%%------------------ 练习题 
%%%%------------------


%%%%------------------
%%%%------------------ 练习题 
\vskip 0.44271\textheight


\par\noindent \textbf{109} \quad 要设计一个容量为$V_0$的长方体开口水箱, 
问水箱长、宽、高等于多少时, 所用材料最省?
\par\noindent \textbf{ 解答}



%%%%------------------ 练习题 
%%%%------------------


%%%%------------------
%%%%------------------ 练习题 
\vskip 0.44271\textheight


\newpage
\section{第十章\quad 重积分}
\subsection{第一节\quad 二重积分概念}
\subsubsection{知识点}
\subsubsection{练习题}
\par\noindent \textbf{110} \quad 函数
$$ f\,(x,y)=\frac{1}{x-y}$$
在
\begin{equation*}    D:
 \begin{cases}
    0\leq x\leq 1 &   \\
    0\leq y\leq 1 & 
 \end{cases}                
\end{equation*}
上二重积分 \hfill (\quad\quad\quad)
\par\noindent \textbf{A} \quad 存在
\par\noindent \textbf{B} \quad 不存在



%%%%------------------ 练习题 
%%%%------------------


%%%%------------------
%%%%------------------ 练习题 
\vskip 0.44271\textheight


\par\noindent \textbf{111} \quad 函数
$$ f\,(x,y)=\frac{x\,^2-y\,^2}{x-y}$$
在
\begin{equation*}    D:
 \begin{cases}
    0\leq x\leq 1 &   \\
    0\leq y\leq 1 & 
 \end{cases}                
\end{equation*}
上二重积分 \hfill (\quad\quad\quad)
\par\noindent \textbf{A} \quad 存在
\par\noindent \textbf{B} \quad 不存在



%%%%------------------ 练习题 
%%%%------------------


%%%%------------------
%%%%------------------ 练习题 
\vskip 0.44271\textheight


\par\noindent \textbf{112} \quad 判断
$$ \left|\iint_D f\,(x,y)\,d\,\sigma\right| \leq \iint_D \big |f\,(x,y)\big |\,d\,\sigma$$
 \hfill (\quad\quad\quad)
\par\noindent \textbf{A} \quad 正确
\par\noindent \textbf{B} \quad 错误



%%%%------------------ 练习题 
%%%%------------------


%%%%------------------
%%%%------------------ 练习题 
\vskip 0.44271\textheight


\par\noindent \textbf{113} \quad 判断
$$ \iint_D (x+y)^2\,d\,\sigma \leq \iint_D (x+y)^3\,d\,\sigma$$
其中$D: (x-2)^2+(y-1)^2\leq 2$ \hfill (\quad\quad\quad)
\par\noindent \textbf{A} \quad 正确
\par\noindent \textbf{B} \quad 错误



%%%%------------------ 练习题 
%%%%------------------


%%%%------------------
%%%%------------------ 练习题 
\vskip 0.44271\textheight


\par\noindent \textbf{114} \quad 判断下列积分的符号
$$ \iint_D \sqrt[3]{1-x\,^2-y\,^2}\,d\,x\,d\,y$$
其中$D: x\,^2+y\,^2\leq 4$ \hfill (\quad\quad\quad)
\par\noindent \textbf{A} \quad 正号
\par\noindent \textbf{B} \quad 负号



%%%%------------------ 练习题 
%%%%------------------


%%%%------------------
%%%%------------------ 练习题 
\vskip 0.44271\textheight


\par\noindent \textbf{115} \quad 计算
$$ I=\int_0^{\frac{\pi}{2}}\int_0^{\frac{\pi}{2}}\sin(x+y)\,d\,x\,d\,y$$
\par\noindent \textbf{ 解答}



%%%%------------------ 练习题 
%%%%------------------


%%%%------------------
%%%%------------------ 练习题 
\vskip 0.44271\textheight


\par\noindent \textbf{116} \quad 估计
$$ \iint_D\,\frac{d\sigma}{\sqrt{x^2+y^2+2xy+16}}$$
的值, 其中$D: 0\leq x\leq 1$, $0\leq y\leq 2$
\par\noindent \textbf{ 解答}



%%%%------------------ 练习题 
%%%%------------------


%%%%------------------
%%%%------------------ 练习题 
\vskip 0.44271\textheight


\par\noindent \textbf{117} \quad 判断下列积分的符号
$$ \iint_D \ln \,\left(x\,^2+y\,^2\,\right)\,d\,x\,d\,y$$
其中$D: a\leq |\,x\,|+|\,y\,|\leq 1$, $0<a<1$
\par\noindent \textbf{ 解答}



%%%%------------------ 练习题 
%%%%------------------


%%%%------------------
%%%%------------------ 练习题 
\vskip 0.44271\textheight


\newpage
\subsection{第二节\quad 二重积分的计算(1)(直角坐标系)}
\subsubsection{知识点}
\subsubsection{练习题}
\par\noindent \textbf{118} \quad 计算
$$ \iint_D \left(x\,^2+y\,^2\,\right)\,d\,\sigma$$
其中$D: |\,x\,|\leq 1, |\,y\,|\leq 1$
\par\noindent \textbf{ 解答}



%%%%------------------ 练习题 
%%%%------------------


%%%%------------------
%%%%------------------ 练习题 
\vskip 0.44271\textheight


\par\noindent \textbf{119} \quad 计算
$$ \iint_D \left(\,x\,^3+3\,x\,^2y+y\,^3\,\right)\,d\,\sigma$$
其中$D: 0\leq \,x\,\leq 1$, $0\leq \,y\,\leq 1$
\par\noindent \textbf{ 解答}



%%%%------------------ 练习题 
%%%%------------------


%%%%------------------
%%%%------------------ 练习题 
\vskip 0.44271\textheight


\par\noindent \textbf{120} \quad 计算
$$ \iint_D \,x\,\sqrt{y\,}\,d\,\sigma$$
其中$D$是由两条抛物线$y=\sqrt{x\,}$, $y=x\,^2$所围成的闭区域
\par\noindent \textbf{ 解答}



%%%%------------------ 练习题 
%%%%------------------

%% %% Ex081 - Ex120



%%%%------------------
%%%%------------------ 练习题 
\vskip 0.44271\textheight


\par\noindent \textbf{121} \quad 计算
$$ \iint_D \,x\,y\,^2\,d\,\sigma$$
其中$D$是由园周$x\,^2+y\,^2=4$及$y$轴所围成的右半闭区域
\par\noindent \textbf{ 解答}



%%%%------------------ 练习题 
%%%%------------------

%%%%------------------
%%%%------------------ 练习题 
\vskip 0.44271\textheight


\par\noindent \textbf{122} \quad 交换积分次序
$$ \int_0^1 d\,y\int_0^y f\,(x,y)\,d\,x$$
\par\noindent \textbf{ 解答}



%%%%------------------ 练习题 
%%%%------------------

%%%%------------------
%%%%------------------ 练习题 
\vskip 0.44271\textheight


\par\noindent \textbf{123} \quad 交换积分次序
$$ \int_0^1 d\,y\int_{-\sqrt{1\,-\,y\,^2}}^{\sqrt{1\,-\,y\,^2}} f\,(\,x,y\,)\,d\,x$$
\par\noindent \textbf{ 解答}



%%%%------------------ 练习题 
%%%%------------------

%%%%------------------
%%%%------------------ 练习题 
\vskip 0.44271\textheight


\newpage
\par\noindent \textbf{第二节\quad 二重积分的计算(2)(极坐标系)} 
\par\noindent \textbf{124} \quad 计算
$$ \iint_D \,e\,^{-x\,^2-y\,^2}\,d\,\sigma$$
其中$D: x\,^2+y\,^2\leq a\,^2$, $a>0$
\par\noindent \textbf{ 解答}



%%%%------------------ 练习题 
%%%%------------------

%%%%------------------
%%%%------------------ 练习题 
\vskip 0.44271\textheight


\par\noindent \textbf{125} \quad 计算
$$ \int_0^{+\infty} \,e\,^{-x\,^2}\,d\,x$$
\par\noindent \textbf{ 解答}



%%%%------------------ 练习题 
%%%%------------------

%%%%------------------
%%%%------------------ 练习题 
\vskip 0.44271\textheight


\par\noindent \textbf{126} \quad 计算
$$ \int_0^{2\,a}\,d\,x \int_0^{\sqrt{\,2\,a\,x\,-\,x\,^2}} \left(x\,^2+y\,^2\right)\,d\,y$$
\par\noindent \textbf{ 解答}



%%%%------------------ 练习题 
%%%%------------------

%%%%------------------
%%%%------------------ 练习题 
\vskip 0.44271\textheight


\par\noindent \textbf{127} \quad 计算
$$ \int_0^1\,d\,x \int_{x\,^2}^x \left(x\,^2+y\,^2\right)^{-\frac{1}{2}}\,d\,y$$
\par\noindent \textbf{ 解答}



%%%%------------------ 练习题 
%%%%------------------

%%%%------------------
%%%%------------------ 练习题 
\vskip 0.44271\textheight


\par\noindent \textbf{128} \quad 计算
$$ \iint_D e\,^{x\,^2+y\,^2}\,d\,\sigma$$
其中$D$是由圆周$x\,^2+y\,^2=4$所围成的闭区域。
\par\noindent \textbf{ 解答}



%%%%------------------ 练习题 
%%%%------------------

%%%%------------------
%%%%------------------ 练习题 
\vskip 0.44271\textheight


\par\noindent \textbf{129} \quad 计算
$$ \iint_D \ln \left(1+x\,^2+y\,^2\right)\,d\,\sigma$$
其中$D$是由圆周$x\,^2+y\,^2=1$及坐标轴所围成的在第一象限内的闭区域。
\par\noindent \textbf{ 解答}



%%%%------------------ 练习题 
%%%%------------------

%%%%------------------
%%%%------------------ 练习题 
\vskip 0.44271\textheight


\par\noindent \textbf{130} \quad 计算
$$ \iint_D \frac{x\,^2}{y\,^2}\,d\,\sigma$$
其中$D$是由直线$x=2$, $y=x$及曲线$x\,y=1$所围成的闭区域。
\par\noindent \textbf{ 解答}



%%%%------------------ 练习题 
%%%%------------------

%%%%------------------
%%%%------------------ 练习题 
\vskip 0.44271\textheight


\par\noindent \textbf{131} \quad 计算
$$ \iint_D \sqrt{x\,^2+y\,^2}\,d\,\sigma$$
其中$D$是圆形环闭区域$a\,^2\leq x\,^2+y\,^2 \leq b\,^2$, $a>0$, $b>0$
\par\noindent \textbf{ 解答}



%%%%------------------ 练习题 
%%%%------------------

%%%%------------------
%%%%------------------ 练习题 
\vskip 0.44271\textheight

\newpage
\par\noindent \textbf{第二节\quad 二重积分的计算(3)(换元法)} 
\par\noindent \textbf{132} \quad 计算
$$ \iint_D \left(\frac{x\,^2}{a\,^2}+\frac{y\,^2}{b\,^2}\right)\,d\,\sigma$$
其中$D$是椭圆形闭区域$\displaystyle\frac{x\,^2}{a\,^2}+\frac{y\,^2}{b\,^2} \leq 1$
\par\noindent \textbf{ 解答}



%%%%------------------ 练习题 
%%%%------------------

%%%%------------------
%%%%------------------ 练习题 
\vskip 0.44271\textheight


\par\noindent \textbf{133} \quad 计算
$$ \iint_D \left(x\,y+\cos x\,\sin y\right)\,d\,\sigma$$
其中$D$是闭区域$-a\leq x\leq a$, $x\leq y\leq a$, $a>0$
\par\noindent \textbf{ 解答}



%%%%------------------ 练习题 
%%%%------------------

%%%%------------------
%%%%------------------ 练习题 
\vskip 0.44271\textheight


\par\noindent \textbf{134} \quad 计算
$$ \iint_D \left(\,y\,^2+3\,x-6\,y+9\,\right)\,d\,\sigma$$
其中$D$是闭区域$x\,^2+y\,^2\leq R\,^2$, $R>0$
\par\noindent \textbf{ 解答}



%%%%------------------ 练习题 
%%%%------------------

%%%%------------------
%%%%------------------ 练习题 
\vskip 0.44271\textheight


\par\noindent \textbf{135} \quad 计算
$$ \iint_D \left|y-x\,^2\,\right|\,d\,\sigma$$
其中$D$是闭区域$0\leq x\leq 1$, $0\leq y\leq 1$
\par\noindent \textbf{ 解答}



%%%%------------------ 练习题 
%%%%------------------

%%%%------------------
%%%%------------------ 练习题 
\vskip 0.44271\textheight


\par\noindent \textbf{136} \quad 计算
$$ \iint_D \left(x\,^2-y\,^2\right)\,d\,\sigma$$
其中$D$是闭区域$0\leq x\leq \pi$, $0\leq y\leq \sin x$
\par\noindent \textbf{ 解答}



%%%%------------------ 练习题 
%%%%------------------

%%%%------------------
%%%%------------------ 练习题 
\vskip 0.44271\textheight


\par\noindent \textbf{138} \quad 设$f\,(x)$为连续函数, $f(2)=1$,
$$ F\,(t)=\int_1^td\,y\int_y^t f\,(x)\,d\,x$$
求$F'(2)$
\par\noindent \textbf{ 解答}



%%%%------------------ 练习题 
%%%%------------------

%%%%------------------
%%%%------------------ 练习题 
\vskip 0.44271\textheight


\par\noindent \textbf{139} \quad 
$$ \int_0^ad\,y\int_0^y e^{\,m\,(a-x)}\,f\,(x)d\,x
=
\int_0^a(a-x)\,e\,^{m\,(a-x)}\,f\,(x)\,d\,x$$
\hfill (\quad\quad\quad)
\par\noindent \textbf{A} \quad 正确
\par\noindent \textbf{B} \quad 错误



%%%%------------------ 练习题 
%%%%------------------


%%%%------------------
%%%%------------------ 练习题 
\vskip 0.44271\textheight

\newpage
\subsection{第三节\quad 三重积分(1)(直角坐标)}
\subsubsection{知识点}
\subsubsection{练习题}
\par\noindent \textbf{139a1} \quad 计算
$$ \iiint_{\Omega} xy^2z^3\,d\,x\,d\,y\,d\,z$$
其中$\Omega$是由曲面$z=xy$, 平面$y=x$, $x=1$和$z=0$所围成的闭区域。
\par\noindent \textbf{ 解答}

%%%%------------------ 练习题 
%%%%------------------

%%%%------------------
%%%%------------------ 练习题 
\vskip 0.44271\textheight

\par\noindent \textbf{139a2} \quad 计算
$$ \iiint_{\Omega} \frac{d\,x\,d\,y\,d\,z}{(1+x+y+z)^3}$$
其中$\Omega$是由平面$x=0$, $y=0$, $z=0$和$x+y+z=1$所围成的闭区域。
\par\noindent \textbf{ 解答}

%%%%------------------ 练习题 
%%%%------------------

%%%%------------------
%%%%------------------ 练习题 
\vskip 0.44271\textheight

\par\noindent \textbf{139a3} \quad 计算
$$ \iiint_{\Omega} xz\,d\,x\,d\,y\,d\,z$$
其中$\Omega$是由平面$z=0$, $z=y$, $y=1$和抛物柱面$y=x^2$所围成的闭区域。
\par\noindent \textbf{ 解答}

%%%%------------------ 练习题 
%%%%------------------

%%%%------------------
%%%%------------------ 练习题 
\vskip 0.44271\textheight

\par\noindent \textbf{139a4} \quad 计算
$$ \iiint_{\Omega} z\,d\,x\,d\,y\,d\,z$$
其中$\Omega$是由锥面$z=\sqrt{x^2+y^2}$和$z=1$所围成的闭区域。
\par\noindent \textbf{ 解答}

%%%%------------------ 练习题 
%%%%------------------

%%%%------------------
%%%%------------------ 练习题 
\vskip 0.44271\textheight

\newpage
\par\noindent \textbf{第三节\quad 三重积分(2)(柱面坐标)}
\par\noindent \textbf{139a5} \quad 利用柱面坐标计算
$$ \iiint_{\Omega} z\,d\,x\,d\,y\,d\,z$$
其中$\Omega$是由曲面$z=\sqrt{2-x^2-y^2}$和$z=x^2+y^2$所围成的闭区域。
\par\noindent \textbf{ 解答}

%%%%------------------ 练习题 
%%%%------------------


%%%%------------------
%%%%------------------ 练习题 
\vskip 0.44271\textheight


\par\noindent \textbf{139a6} \quad 计算
$$ \iiint_{\Omega} xy\,d\,v$$
其中$\Omega$是柱面$x^2+y^2=1$及平面$z=1$, $z=0$, $x=0$, $y=0$所围成的在第一卦限内的闭区域。
\par\noindent \textbf{ 解答}

%%%%------------------ 练习题 
%%%%------------------

%%%%------------------
%%%%------------------ 练习题 
\vskip 0.44271\textheight

\par\noindent \textbf{139a7} \quad 计算
$$ \iiint_{\Omega} \left(x^2+y^2\right)\,d\,v$$
其中$\Omega$是由曲面$4z^2=x^2+y^2$及平面$z=1$所围成的闭区域。
\par\noindent \textbf{ 解答}



%%%%------------------ 练习题 
%%%%------------------

%%%%------------------
%%%%------------------ 练习题 
\vskip 0.44271\textheight

\newpage
\par\noindent \textbf{第三节\quad 三重积分(3)(球面坐标)}
\par\noindent \textbf{139a8} \quad 利用球面坐标计算
$$ \iiint_{\Omega} \left(x^2+y^2+z^2\right)\,d\,x\,d\,y\,d\,z$$
其中$\Omega$是由球面$x^2+y^2+z^2=1$所围成的闭区域。
\par\noindent \textbf{ 解答}

%%%%------------------ 练习题 
%%%%------------------

%%%%------------------
%%%%------------------ 练习题 
\vskip 0.44271\textheight


\par\noindent \textbf{139a8b1} \quad 计算
$$ \iiint_{\Omega} \frac{z\,\ln \,(x\,^2+y\,^2+z\,^2+1)}{x\,^2+y\,^2+z\,^2+1}\,d\,v$$
其中$\Omega$是由球面$x\,^2+y\,^2+z\,^2=1$所围成的闭区域。
\par\noindent \textbf{ 解答}



%%%%------------------ 练习题 
%%%%------------------

\newpage
\subsection{第四节\quad 重积分的应用}
\subsubsection{知识点}
\subsubsection{练习题}

\par\noindent \textbf{139a9} \quad 求球面$x^2+y^2+z^2=1$含在圆柱面$x^2+y^2=x$内部的那部分面积。
\par\noindent \textbf{ 解答}

%%%%------------------
%%%%------------------ 练习题 
\vskip 0.44271\textheight

\par\noindent \textbf{139a10} \quad 求锥面$z=\sqrt{x^2+y^2}$被柱面$z^2=2x$所割下部分的曲面面积。
\par\noindent \textbf{ 解答}

%%%%------------------
%%%%------------------ 练习题 
\vskip 0.44271\textheight

\newpage
\section{第十一章\quad 曲线积分与曲面积分}
\subsection{第一节\quad 对弧长曲线积分}
\subsubsection{知识点}
\subsubsection{练习题}
\par\noindent \textbf{140} \quad 计算
$$ \int_L (x+y)\,d\,s$$
其中$L$为连接$(1,0)$及$(0,1)$两点的直线段。
\par\noindent \textbf{ 解答}



%%%%------------------ 练习题 
%%%%------------------

%%%%------------------
%%%%------------------ 练习题 
\vskip 0.44271\textheight


\par\noindent \textbf{141} \quad 计算
$$ \int_L y\,^2\,d\,s$$
其中$L$为摆线的一拱, $x=a\,(t-\sin t\,)$, $y=a\,(1-\cos t\,)$, $0\leq t \leq 2\,\pi$
\par\noindent \textbf{ 解答}



%%%%------------------ 练习题 
%%%%------------------

%%%%------------------
%%%%------------------ 练习题 
\vskip 0.44271\textheight


\par\noindent \textbf{142} \quad 计算
$$ \int_L (x+y)\,d\,s$$
其中$L$为连接$(1,0)$, $(0,1)$两点的直线段。
\par\noindent \textbf{ 解答}



%%%%------------------ 练习题 
%%%%------------------

%%%%------------------
%%%%------------------ 练习题 
\vskip 0.44271\textheight


\newpage
\subsection{第二节\quad 对坐标曲线积分}
\subsubsection{知识点}
\subsubsection{练习题}
\par\noindent \textbf{143} \quad 计算
$$ \int_L (x\,^2-y\,^2)\,d\,x$$
其中$L$是抛物线$y=x\,^2$上从点 $(0,0)$到点 $(2,4)$的一段弧。
\par\noindent \textbf{ 解答}



%%%%------------------ 练习题 
%%%%------------------

%%%%------------------
%%%%------------------ 练习题 
\vskip 0.44271\textheight


\par\noindent \textbf{144} \quad 计算
$$ \oint_L \frac{(x+y)\,d\,x-(x-y)\,d\,y}{x\,^2+y\,^2}$$
其中$L$为圆周$x\,^2+y\,^2=a\,^2$, $a>0$, 逆时针方向。
\par\noindent \textbf{ 解答}



%%%%------------------ 练习题 
%%%%------------------

%%%%------------------
%%%%------------------ 练习题 
\vskip 0.44271\textheight


\par\noindent \textbf{145} \quad 计算
$$ \int_L (x+y\,)\,d\,x+(y-x\,)\,d\,y$$
其中$L$是曲线$x=2t\,^2+t+1$, $y=t\,^2+1$上从点$(1,1)$到点$(4,2)$的一段弧。
\par\noindent \textbf{ 解答}



%%%%------------------ 练习题 
%%%%------------------

%%%%------------------
%%%%------------------ 练习题 
\vskip 0.44271\textheight


\par\noindent \textbf{146} \quad 求下列曲线所围成的图形的面积
$$x=a\cos ^3 t,\quad y=a\sin ^3t$$
\par\noindent \textbf{ 解答}



%%%%------------------ 练习题 
%%%%------------------

%%%%------------------
%%%%------------------ 练习题 
\vskip 0.44271\textheight


\newpage
\subsection{第三节\quad 格林公式}
\subsubsection{知识点}
\subsubsection{练习题}
\par\noindent \textbf{147} \quad 计算
$$ \oint_L \frac{y\,d\,x-x\,d\,y}{2(x\,^2+y\,^2)}$$
其中$L$为圆周$(x-1)\,^2+y\,^2=2$, 逆时针方向。
\par\noindent \textbf{ 解答}



%%%%------------------ 练习题 
%%%%------------------

%%%%------------------
%%%%------------------ 练习题 
\vskip 0.44271\textheight


\par\noindent \textbf{148} \quad 计算
$$ \int_L \left(2\,x\,y\,^3-y\,^2\cos x\,)\,d\,x+(1-2\,y\,\sin x+3\,x\,^2\,y\,^2\,\right)\,d\,y$$
其中$L$为抛物线$2\,x=\pi\, y\,^2$上由点$A\,(0,0)$到$\displaystyle B\,\left(\frac{\pi}{2},1\right)$的一段弧。
\par\noindent \textbf{ 解答}



%%%%------------------ 练习题 
%%%%------------------

%%%%------------------
%%%%------------------ 练习题 
\vskip 0.44271\textheight


\newpage
\subsection{第四节\quad 对面积曲面积分}
\subsubsection{知识点}
\subsubsection{练习题}
\par\noindent \textbf{149} \quad 计算
$$ \int_{\Sigma} \left(\,x\,^2+y\,^2\,\right)\,d\,S$$
其中$\Sigma$是$z\,^2=3\left(x\,^2+y\,^2\right)$被平面$z=0$和$z=3$所截得的部分。
\par\noindent \textbf{ 解答}



%%%%------------------ 练习题 
%%%%------------------

%%%%------------------
%%%%------------------ 练习题 
\vskip 0.44271\textheight


\par\noindent \textbf{150} \quad 计算
$$ \iint_{\Sigma} \left(\,2xy-2x^2-x+z\,\right)\,d\,S$$
其中$\Sigma$是平面$2x+2y+z=6$在第一卦限中的部分。
\par\noindent \textbf{ 解答}



%%%%------------------ 练习题 
%%%%------------------

%%%%------------------
%%%%------------------ 练习题 
\vskip 0.44271\textheight


\par\noindent \textbf{151} \quad 计算
$$ \iint_{\Sigma} \left(\,x\,y+y\,z+z\,x\,\right)\,d\,S$$
其中$\Sigma$是锥面$z=\sqrt{x\,^2+y\,^2}$被柱面$x\,^2+y\,^2=2a\,x$所截得的有限部分。
\par\noindent \textbf{ 解答}



%%%%------------------ 练习题 
%%%%------------------

%%%%------------------
%%%%------------------ 练习题 
\vskip 0.44271\textheight

\newpage
\subsection{第五节\quad 对坐标曲面积分}
\subsubsection{知识点}
\subsubsection{练习题}
\par\noindent \textbf{152} \quad 计算
$$ \iint_{\Sigma} \,x\,^2y\,^2z\,d\,x\,d\,y$$
其中$\Sigma$是球面$x\,^2+y\,^2+z\,^2=R\,^2$的下半部分的下侧。
\par\noindent \textbf{ 解答}



%%%%------------------ 练习题 
%%%%------------------

%%%%------------------
%%%%------------------ 练习题 
\vskip 0.44271\textheight


\par\noindent \textbf{153} \quad 计算
$$\oiint_{\Sigma} \,x\,z\,d\,x\,d\,y+\,x\,y\,d\,y\,d\,z+\,y\,z\,d\,z\,d\,x$$
其中$\Sigma$是平面$x=0$, $y=0$, $z=0$, $x+y+z=1$ 所围成的空间区域的整个边界曲面的外侧。
\par\noindent \textbf{ 解答}



%%%%------------------ 练习题 
%%%%------------------

%%%%------------------
%%%%------------------ 练习题 
\vskip 0.44271\textheight


\par\noindent \textbf{154} \quad 计算
$$\iint_{\Sigma} \,z\,d\,x\,d\,y+\,x\,d\,y\,d\,z+\,y\,d\,z\,d\,x$$
其中$\Sigma$是柱面$x^2+y^2=1$被平面$z=0$, $z=3$所截得的在第一卦限内的部分的前侧。
\par\noindent \textbf{ 解答}



%%%%------------------ 练习题 
%%%%------------------

%%%%------------------
%%%%------------------ 练习题 
\vskip 0.44271\textheight

\newpage
\subsection{第六节\quad 高斯公式}
\subsubsection{知识点}
\subsubsection{练习题}
\par\noindent \textbf{154a1} \quad 计算曲面积分
$$\iint_{\Sigma} \,\left(\,x\,^3+y\,^2\,\right)\,d\,y\,d\,z+\,\left(\,y\,^3+z\,^2\,\right)\,d\,z\,d\,x+\,\left(\,z\,^3+x\,^2\,\right)\,d\,x\,d\,y$$
其中$\Sigma$是上半球面$z=\sqrt{1-x\,^2-y\,^2}$的上侧。
\par\noindent \textbf{ 解答}



%%%%------------------ 练习题 
%%%%------------------

%%%%------------------
%%%%------------------ 练习题 
\vskip 0.44271\textheight


\par\noindent \textbf{155} \quad 计算
$$\oiint_{\Sigma} \,z^2\,d\,x\,d\,y+\,x^2\,d\,y\,d\,z+\,y^2\,d\,z\,d\,x$$
其中$\Sigma$是平面$x=0$, $y=0$, $z=0$, $x=a$, $y=a$, $z=a$所围成的立体的表面的外侧。
\par\noindent \textbf{ 解答}



%%%%------------------ 练习题 
%%%%------------------

%%%%------------------
%%%%------------------ 练习题 
\vskip 0.44271\textheight


\par\noindent \textbf{156} \quad 计算
$$\oiint_{\Sigma} \,x\,z\,^2\,d\,y\,d\,z+\,(x\,^2\,y-z\,^3)\,d\,z\,d\,x+\,(2\,x\,y+y\,^2\,z)\,d\,x\,d\,y$$
其中$\Sigma$是由上半球体
$$0\leq z\leq \sqrt{a\,^2-x\,^2-y\,^2}$$
和圆盘
$$x\,^2+y\,^2\leq a\,^2$$
所围成的封闭立体的外侧面。
\par\noindent \textbf{ 解答}



%%%%------------------ 练习题 
%%%%------------------

%%%%------------------
%%%%------------------ 练习题 
\vskip 0.44271\textheight


\par\noindent \textbf{157} \quad 计算
$$\oiint_{\Sigma} \,4x\,z\,d\,y\,d\,z-\,y\,^2\,d\,z\,d\,x+\,yz\,d\,x\,d\,y$$
其中$\Sigma$是平面$x=0$, $y=0$, $z=0$, $x=1$, $y=1$, $z=1$所围立体的整个外侧面。
\par\noindent \textbf{ 解答}



%%%%------------------ 练习题 
%%%%------------------

%%%%------------------
%%%%------------------ 练习题 
\vskip 0.44271\textheight

\newpage
\section{第十二章\quad 无穷级数}
\subsection{第一节\quad 常数项级数}
\subsubsection{知识点}
\subsubsection{练习题}
\par\noindent \textbf{158} \quad 讨论级数的敛散性
$$\sum_{n=1}^{\infty}\left(\,\sqrt{n+1}-\sqrt{n}\,\right)$$
\par\noindent \textbf{ 解答}



%%%%------------------ 练习题 
%%%%------------------

%%%%------------------
%%%%------------------ 练习题 
\vskip 0.44271\textheight


\par\noindent \textbf{159} \quad 讨论级数的敛散性
$$\sum_{n=1}^{\infty}\frac{1}{(2n-1)(2n+1)}$$
\par\noindent \textbf{ 解答}



%%%%------------------ 练习题 
%%%%------------------

%%%%------------------
%%%%------------------ 练习题 
\vskip 0.44271\textheight


\par\noindent \textbf{160} \quad 讨论级数的敛散性
$$\sum_{n=1}^{\infty}\sin\frac{n\,\pi}{6}$$
\par\noindent \textbf{ 积化和差公式}
\par\noindent \textbf{ 解答}



%%%%------------------ 练习题 
%%%%------------------

%%%%------------------
%%%%------------------ 练习题 
\vskip 0.44271\textheight

\newpage
\subsection{第二节\quad 数项级数及审敛法}
\subsubsection{知识点}
\subsubsection{练习题}
\par\noindent \textbf{161} \quad 讨论级数的敛散性
$$\sum_{n=1}^{\infty}\frac{1}{2n-1}$$
\par\noindent \textbf{ 解答}



%%%%------------------ 练习题 
%%%%------------------

%%%%------------------
%%%%------------------ 练习题 
\vskip 0.44271\textheight


\par\noindent \textbf{162} \quad 讨论级数的敛散性
$$\sum_{n=1}^{\infty}\frac{1}{(n+1)(n+4)}$$
\par\noindent \textbf{ 解答}



%%%%------------------ 练习题 
%%%%------------------

%%%%------------------
%%%%------------------ 练习题 
\vskip 0.44271\textheight


\par\noindent \textbf{163} \quad 讨论级数的敛散性
$$\sum_{n=1}^{\infty}\frac{1}{1+a\,^n}, \quad a>0$$
\par\noindent \textbf{ 解答}



%%%%------------------ 练习题 
%%%%------------------

%%%%------------------
%%%%------------------ 练习题 
\vskip 0.44271\textheight


\par\noindent \textbf{164} \quad 讨论级数的敛散性
$$\sum_{n=1}^{\infty}\frac{2\,^n\cdot n\,!}{n\,^n}$$
\par\noindent \textbf{ 解答}



%%%%------------------ 练习题 
%%%%------------------

%%%%------------------
%%%%------------------ 练习题 
\vskip 0.44271\textheight


\par\noindent \textbf{165} \quad 讨论级数的敛散性
$$\sum_{n=1}^{\infty}\left(\frac{n}{2n+1}\right)^n$$
\par\noindent \textbf{ 解答}



%%%%------------------ 练习题 
%%%%------------------

%%%%------------------
%%%%------------------ 练习题 
\vskip 0.44271\textheight


\par\noindent \textbf{166} \quad 讨论级数的敛散性
$$\sum_{n=1}^{\infty} 2\,^n\sin\frac{\pi}{3\,^n}$$
\par\noindent \textbf{ 解答}



%%%%------------------ 练习题 
%%%%------------------

%%%%------------------
%%%%------------------ 练习题 
\vskip 0.44271\textheight

\newpage
\par\noindent \textbf{第二节\quad 数项级数及审敛法(续)(交错级数的审敛法)}
\par\noindent \textbf{167} \quad 讨论级数的敛散性
$$\sum_{n=1}^{\infty} \,(-1)\,^{n-1} \frac{n}{3\,^{n-1}}$$
\par\noindent \textbf{ 解答}



%%%%------------------ 练习题 
%%%%------------------

%%%%------------------
%%%%------------------ 练习题 
\vskip 0.44271\textheight


\par\noindent \textbf{168} \quad 讨论级数的敛散性
$$\sum_{n=1}^{\infty} \,(-1)\,^{n-1} \,\frac{1}{\ln\,(n+1)}$$
\par\noindent \textbf{ 解答}



%%%%------------------ 练习题 
%%%%------------------

%%%%------------------
%%%%------------------ 练习题 
\vskip 0.44271\textheight


\par\noindent \textbf{169} \quad 讨论级数的敛散性
$$\sum_{n=1}^{\infty} \,(-1)\,^{n+1} \,\frac{2\,^{n\,^2}}{n\,!}$$
\par\noindent \textbf{ 解答}



%%%%------------------ 练习题 
%%%%------------------

%%%%------------------
%%%%------------------ 练习题 
\vskip 0.44271\textheight


\par\noindent \textbf{169a1} \quad 求极限
$$\lim_{n\rightarrow\infty}\,\frac{5\,^n}{2\,^n\cdot n\,!}$$
\par\noindent \textbf{ 解答}



%%%%------------------ 练习题 
%%%%------------------

%%%%------------------
%%%%------------------ 练习题 
\vskip 0.44271\textheight


\par\noindent \textbf{169a2} \quad 讨论级数的敛散性
$$\sum_{n=1}^{\infty} \,\frac{(-1)\,^n }{n-\ln n}$$
\par\noindent \textbf{ 解答}



%%%%------------------ 练习题 
%%%%------------------

%%%%------------------
%%%%------------------ 练习题 
\vskip 0.44271\textheight

\newpage
\subsection{第三节\quad 幂级数}
\subsubsection{知识点}
\subsubsection{练习题}
\par\noindent \textbf{170} \quad 求下列级数的收敛区间
$$\sum_{n=1}^{\infty} \,(-1)\,^n \,\frac{x\,^{2n+1}}{2n+1}$$
\par\noindent \textbf{ 解答}



%%%%------------------ 练习题 
%%%%------------------

%%%%------------------
%%%%------------------ 练习题 
\vskip 0.44271\textheight


\par\noindent \textbf{171} \quad 求下列级数的收敛区间
$$\sum_{n=1}^{\infty} \,\frac{(x-5)^n}{\sqrt{n}}$$
\par\noindent \textbf{ 解答}



%%%%------------------ 练习题 
%%%%------------------

%%%%------------------
%%%%------------------ 练习题 
\vskip 0.44271\textheight


\par\noindent \textbf{172} \quad 求下列级数的和函数
$$\sum_{n=1}^{\infty} \,n\,x\,^{n-1}$$
\par\noindent \textbf{ 解答}



%%%%------------------ 练习题 
%%%%------------------

%%%%------------------
%%%%------------------ 练习题 
\vskip 0.44271\textheight

\newpage
\subsection{第四节\quad 函数展成幂级数}
\subsubsection{知识点}
\subsubsection{练习题}
\par\noindent \textbf{173} \quad 将双曲正弦函数展开成$x$的幂级数。
\par\noindent \textbf{ 解答}



%%%%------------------ 练习题 
%%%%------------------

%%%%------------------
%%%%------------------ 练习题 
\vskip 0.44271\textheight


\par\noindent \textbf{173a} \quad 将下列函数展开成$x$的幂级数
$$\frac{e^x-e^{-x}}{2}$$
\par\noindent \textbf{ 解答}



%%%%------------------ 练习题 
%%%%------------------

%%%%------------------
%%%%------------------ 练习题 
\vskip 0.44271\textheight


\par\noindent \textbf{174} \quad 将下列函数展开成$x$的幂级数
$$\frac{x}{\sqrt{1+x\,^2}}$$
\par\noindent \textbf{ 解答}



%%%%------------------ 练习题 
%%%%------------------

%%%%------------------
%%%%------------------ 练习题 
\vskip 0.44271\textheight

\newpage
\par\noindent \textbf{第四节\quad 函数展成幂级数(续)}
\par\noindent \textbf{175} \quad 将下列函数展开成$(x-1)$的幂级数
$$\log x$$
\par\noindent \textbf{ 解答}



%%%%------------------ 练习题 
%%%%------------------

%%%%------------------
%%%%------------------ 练习题 
\vskip 0.44271\textheight


\par\noindent \textbf{176} \quad 将下列函数展开成$(x+4)$的幂级数
$$\frac {1}{x\,^2+3\,x+2}$$
\par\noindent \textbf{ 解答}



%%%%------------------ 练习题 
%%%%------------------

%%%%------------------
%%%%------------------ 练习题 
\vskip 0.44271\textheight


\par\noindent \textbf{177} \quad 将下列函数展开成$x$的幂级数
$$e\,^x\,\cos x$$
\par\noindent \textbf{ 借助欧拉公式}
\par\noindent \textbf{ 解答}



%%%%------------------ 练习题 
%%%%------------------
%%%%------------------ 练习题 
%%%%------------------
%%
\newpage
%%
%%%%%%%%%%%%%%%%%%%%%%%
%%
%%%%%%%%%%%%%%%%%%%%%%%%%%%%%%%%%%%%%%%%%%%%%%%%%%%%%%%%%%%%%%%%%%%%%
%% 文献索引总目录
\bibliography{F:/git/secret/d/dt/BD5241BD-F245-4D33-8EDC-99FFCCF722A4.bib}
%1. \href{http://ds.iris.edu/ds/nodes/dmc/data/}{iris},Incorporated Research Institutions for Seismology (IRIS), 数据集.\par
\end{document}

%%%%%%%%%%%%%%%%%%%%%%%%%%%%%%%%%%%%%%%%%%%%%%%%%%%%%%%%%%%%%%%%%%%%%%
%%% 5f73ca99-9b5f-4beb-b404-c67ac8665ede.tex ends here
